% !TeX encoding = UTF-8

% 载入 SJTUThesis 模版
\documentclass[type=bachelor,oneside]{sjtuthesis}
% 选项
%   type=[doctor|master|bachelor],     % 可选(默认:master),论文类型
%   zihao=[-4|5],                      % 可选(默认:-4),正文字号大小
%   lang=[zh|en|de|ja],                % 可选(默认:zh),论文的主要语言
%   review,                            % 可选(默认:关闭),盲审模式
%   [twoside|oneside],                 % 可选(默认:twoside),双页或单页边距模式
%   [openright|openany],               % 可选(默认:openright),奇数页或任意页开始新章
%   math-style=[ISO|TeX],              % 可选 (默认:ISO),数学符号样式
\usepackage[hidelinks]{hyperref}


% 论文基本配置,加载宏包等全局配置
% !TEX root = ./main.tex

\sjtusetup{
  %
  %******************************
  % 注意:
  %   1. 配置里面不要出现空行
  %   2. 不需要的配置信息可以删除
  %******************************
  %
  % 信息录入
  %
  info = {%
    %
    % 标题
    %
    zh / title           = {基于推理模拟器的LLM全局调度方法研究},
    en / title           = {Research on Global Scheduling Methods for Large Language Models Based on an Inference Simulator
},
    %
    % 标题页标题
    %   可使用“\\”命令手动控制换行
    %
    % zh / display-title   = {上海交通大学学位论文\\ \LaTeX{} 模板示例文档},
    % en / display-title   = {A Sample Document \\ for \LaTeX-based SJTU Thesis Template},
    %
    % 关键词
    %
    zh / keywords        = {推理模拟器,调度决策,吞吐优化},
    en / keywords        = {Inference Simulator, Scheduling Strategy, throughput optimization},
    %
    % 姓名
    %
    zh / author          = {张铠玺},
    en / author          = {\textbf{Zhang Kaixi}},
    %
    % 指导教师
    %
    zh / supervisor      = {魏星达},
    en / supervisor      = {\textbf{Wei Xingda}},
    %
    % 副指导教师
    %
    % assoc-supervisor  = {某某教授},
    % assoc-supervisor* = {Prof. Uom Uom},
    %
    % 学号
    %
    id              = {522031910433},
    %
    % 学位
    %   本科生不需要填写
    %
    zh / degree          = {学士},
    en / degree          = {Master of Engineering},
    %
    % 专业
    %
    zh / major           = {软件工程},
    en / major           = {Software Engineering},
    %
    % 所属院系
    %
    zh / department      = {计算机学院},
    en / department      = {School of Computer Science},
    %
    % 答辩日期
    %   使用 ISO 格式 (yyyy-mm-dd);默认为当前时间
    %
     date                 = {2025-06},
    %
    % 标题页显示日期
    %   覆盖对应标题页的日期显示,原样输出
    %
     zh / display-date    = {2026 年 06 月},
    %
    % 资助基金
    %
    % zh / fund  = {
    %                {国家 973 项目 (No. 2025CB000000)},
    %                {国家自然科学基金 (No. 81120250000)},
    %              },
    % en / fund  = {
    %                {National Basic Research Program of China (Grant No. 2025CB000000)},
    %                {National Natural Science Foundation of China (Grant No. 81120250000)},
    %              },
  },
  %
  % 风格设置
  %
  style = {%
    %
    % 论文标题页 logo 颜色 (red/blue/black)
    %
    % title-logo-color = black,
  },
  %
  % 名称设置
  %
  name = {
    % bib             = {References},
    % ack             = {谢\hspace{\ccwd}辞},
    % achv            = {攻读学位期间完成的论文},
    digest={},
  },
}

% 使用 BibLaTeX 处理参考文献
%   biblatex-gb7714-2015 常用选项
%     gbnamefmt=lowercase     姓名大小写由输入信息确定
%     gbpub=false             禁用出版信息缺失处理
\usepackage[backend=biber,style=gb7714-2015]{biblatex}
% 文献表字体
% \renewcommand{\bibfont}{\zihao{5}\fixedlineskip{15.6bp}}
% 文献表条目间的间距
\setlength{\bibitemsep}{0pt}
% 导入参考文献数据库
\addbibresource{refs.bib}

% 脚注格式
\usepackage[perpage,bottom,hang]{footmisc}

% 定义图片文件目录与扩展名
\graphicspath{{figures/}}
\DeclareGraphicsExtensions{.pdf,.eps,.png,.jpg,.jpeg}

% 确定浮动对象的位置,可以使用 [H],强制将浮动对象放到这里(可能效果很差)
% \usepackage{float}

% 固定宽度的表格
% \usepackage{tabularx}

% 使用三线表:toprule,midrule,bottomrule。
\usepackage{booktabs}

% 表格中支持跨行
\usepackage{multirow}

% 表格中数字按小数点对齐
\usepackage{dcolumn}
\newcolumntype{d}[1]{D{.}{.}{#1}}

% 使用长表格
\usepackage{longtable}

% 附带脚注的表格
\usepackage{threeparttable}

% 附带脚注的长表格
\usepackage{threeparttablex}

% 算法环境宏包
\usepackage[ruled,vlined,linesnumbered]{algorithm2e}
% \usepackage{algorithm, algorithmicx, algpseudocode}

% 代码环境宏包
\usepackage{listings}
\lstdefinestyle{lstStyleCode}{%
  aboveskip         = \medskipamount,
  belowskip         = \medskipamount,
  basicstyle        = \ttfamily\zihao{6},
  commentstyle      = \slshape\color{black!60},
  stringstyle       = \color{green!40!black!100},
  keywordstyle      = \bfseries\color{blue!50!black},
  extendedchars     = false,
  upquote           = true,
  tabsize           = 2,
  showstringspaces  = false,
  xleftmargin       = 1em,
  xrightmargin      = 1em,
  breaklines        = false,
  framexleftmargin  = 1em,
  framexrightmargin = 1em,
  backgroundcolor   = \color{gray!10},
  columns           = flexible,
  keepspaces        = true,
  texcl             = true,
  mathescape        = true
}
\lstnewenvironment{codeblock}[1][]{%
  \lstset{style=lstStyleCode,#1}}{}

% 直立体数学符号
\providecommand{\dd}{\mathop{}\!\mathrm{d}}
\providecommand{\ee}{\mathrm{e}}
\providecommand{\ii}{\mathrm{i}}
\providecommand{\jj}{\mathrm{j}}

% 国际单位制宏包
\usepackage{siunitx}

% 定理环境宏包
\usepackage{ntheorem}
% \usepackage{amsthm}

% 绘图宏包
\usepackage{tikz}
\usetikzlibrary{arrows.meta, shapes.geometric}

% 一些文档中用到的 logo
\usepackage{hologo}
\providecommand{\XeTeX}{\hologo{XeTeX}}
\providecommand{\BibLaTeX}{\textsc{Bib}\LaTeX}

% 借用 ltxdoc 里面的几个命令方便写文档
\DeclareRobustCommand\cs[1]{\texttt{\char`\\#1}}
\providecommand\pkg[1]{{\sffamily#1}}

% hyperref 宏包在最后调用
\usepackage{hyperref}

% E-mail
\providecommand{\email}[1]{\href{mailto:#1}{\urlstyle{tt}\nolinkurl{#1}}}


\begin{document}
\setlength{\baselineskip}{20pt}

%TC:ignore

% 标题页
\maketitle

% 原创性声明及使用授权书
\copyrightpage
% 插入外置原创性声明及使用授权书
% 此时必须在导言区使用 \usepackage{pdfpages}
% \copyrightpage[scans/sample-copyright.pdf]

% 前置部分
\frontmatter
{
\fancyhead[LE,RO]{}
{
\ctexset{chapter={afterskip=26bp}}
% 摘要
% !TEX root = ../main.tex

\begin{abstract}[zh]
\addcontentsline{toc}{chapter}{摘 \quad 要}
  学位论文是本科生从事科研工作的成果的主要表现,集中表明了作者在研究工作中获得的新的发明、理论或见解,也是科研领域中的重要文献资料和社会的宝贵财富。
 \par 为了提高本科生学位论文的质量,做到学位论文在内容和格式上的规范化与统一化,特制作本模板。
\end{abstract}

{
\newfontface{\arial}{Arial}[Scale=0.94]
\ctexset{chapter/format+={\arial}}
\begin{abstract}[en]
\addcontentsline{toc}{chapter}{ABSTRACT}
As a primary means of demonstrating research findings for undergraduate students, dissertation is a systematic and standardized record of the new inventions, theories or insights obtained by the author in the research work. It can not only function as an important reference when students pursue further studies, but also contribute to scientific research and social development.
\par This template is therefore made to improve the quality of undergraduates’ dissertation and to further standardize it both in content and in format.

\end{abstract}
}
}

{
\ctexset{chapter={afterskip=26bp}}
\renewcommand{\cftchapfont}{\zihao{4}\bfseries}
\renewcommand{\cftsecfont}{\zihao{-4}}
\renewcommand{\cftsubsecfont}{\zihao{5}}
% 目录
\tableofcontents
}
}
% % 插图索引
% \listoffigures*
% % 表格索引
% \listoftables*
% % 算法索引
% \listofalgorithms*
% % 符号对照表
% \input{contents/nomenclature}

%TC:endignore

% 主体部分
\mainmatter

% 正文内容
{
\ctexset{chapter={afterskip=26bp}}
% !TEX root = ../main.tex

\chapter{绪论}
\label{ch:intro}

\section{背景}

随着大语言模型(Large Language Models,LLMs)的涌现能力在智能对话、代码生成以及智能体系统等应用场景中的广泛验证,各领域对大语言模型的调用需求呈几何倍数增长。然而用于大模型部署的硬件资源价格非常昂贵,绝大多数使用大模型的用户、企业难以独立部署所需的LLM推理服务。这为提供模型即服务(Model-as-a-Service, MaaS)的大模型厂商带来了飞速增长的模型调用需求。如何面向用户提供高效的推理服务成为当前AI系统架构的重要问题。

\section{LLM服务厂商部署方案}

\subsection{集群部署形态}

在模型即服务厂商的实际部署中,为了满足用户海量请求调用对硬件和算力的需求,LLM推理服务通常以集群形式运行。通过复制推理实例的方式可以高效、方便地扩展硬件来提高服务能力,每个推理实例作为独立的模型部署单元,通过批处理管理算力资源。在推理实例之上,全局调度器(global scheduler)将到达集群的调用请求路由至合适的推理实例执行。这对交互型应用如ChatGPT、copilots提高用户体验非常重要,也往往作为大模型服务厂商衡量集群服务能力和向用户承诺请求时延的重要指标。

\subsection{Prefill/Decode 共置部署形态}

在LLM推理服务中,请求的计算过程可分为Prefill与Decode两个阶段。Prefill阶段处理输入token并生成第一个输出token,计算密集但执行次数少;Decode阶段逐个生成后续token,访存密集且持续执行。在PD共置部署形态下,这两个阶段在同一推理实例、同一GPU资源池上混合执行,实例调度器在每个调度步中同时处理部分Prefill与Decode请求,调度形成混合批(Mixed Batch)。

两阶段在GPU上竞争计算资源与显存带宽,使得PD共置场景下实例负载与TTFT/TPOT的关联更加复杂。简单的负载指标往往难以准确反映实例的服务能力空余,调度决策更依赖对实例当前状态与未来批处理行为的综合判断。相比之下,PD分离场景将Prefill与Decode分别放在不同的GPU实例上执行,可以分别进行批处理与容量规划,负载刻画相对简单。

尽管PD分离近年来受到广泛关注,但当前大模型服务厂商与开源系统中仍存在PD共置的部署。PD分离需要额外的系统改造与跨机通信,包括请求状态迁移、KVCache传递或重计算等,对工程复杂度与部署稳定性提出更高要求。在一定规模与成本约束下,PD共置依然是实际可落地、可维护的部署方案。因此,面向PD共置场景研究调度策略仍然具有现实价值。

\section{性能模拟器与在线预测}

\subsection{LLM性能模拟器概述}

LLM性能模拟器是分析和评估AI服务系统的重要工具,通常通过在CPU上模拟大规模GPU集群的执行过程,在给定系统配置和请求trace的情况下离线得到性能指标。根据面对的场景和特点,可以将其分为两类。

训练性能模拟器主要面向训练场景,较有影响力的工作如SimAI,特点是使用重量级的网络模拟和较为简单的计算时长模拟。训练场景由于数据并行的存在会产生大量通信需求和网络开销;而对于计算,训练时的批处理模式较为固定,同一批中的所有序列长度保持相同,使得计算建模相对简单。

离线推理模拟器面向推理场景,目标是模拟集群吞吐、优化集群配置。推理时不存在数据并行,网络建模较为简单,但批处理和分块预填充等行为导致批处理的计算时延存在动态性,计算建模难度更高。

从实现角度看,模拟器可以拆分为两个核心组件。调度器(Scheduler)维护推理实例的状态信息,例如KVCache、请求队列等,通过模拟推理实例的请求调度行为,按照步进的方式构造Batch。由于常见实例调度器的主要行为是确定性的(例如vLLM),这是模拟器能够模拟实例调度行为的前提,保证了能够得到与真实实例调度相同的Batch构造结果。第二个部分是预测器(Predictor),对给定的Batch预测执行时间,通过对当前批处理的特征进行建模来得到对应的执行时间。已有工作通过不同的性能模型来建模批处理的执行时间,例如线性分析模型、神经网络、随机森林等。如何实现准确、高效的性能模型需要对于预测特征进行仔细的选取和设计,但这不是本文的讨论范围,本文的预测器实现参考当前使用较为广泛的推理模拟器系统Vidur,用于验证在线模拟器设计和优化方法的有效性。

\subsection{调度策略的两类方法}

在集群部署的全局调度器中,合理的调度策略能够通过负载均衡充分利用不同实例的计算资源,还能通过前缀匹配重用键值缓存避免重复计算,从而降低请求推理时延、提高系统吞吐。实现一个高效、兼顾负载均衡与前缀匹配的调度策略并不容易,已有的调度实现可以按照如下方式分类。

第一类是基于历史和当前服务指标的混合类方法。这类方法通过服务指标(例如Batch size、Total tokens)来描述实例负载和服务情况,通过加权组合或过滤进行筛选。这是最直接、简单的调度策略实现,能够通过调整混合参数、筛选阈值来控制对负载均衡、前缀匹配的选择权重,在工业界和学术界的服务系统中都有部署。但这类方法存在两个局限:首先,这些指标难以捕捉推理实例中真实、全面的负载情况,尤其在PD共置下Prefill与Decode的相互影响未被显式建模;其次,多个指标混合时的系数难以确定,往往通过网格搜索或人类专家启发式调优的方式设置。由于难以调参,次优的参数配置可能无法完全发掘推理系统的服务能力。代表系统包括vLLM(负载均衡优先)、ai-Dynamo和Company-X(线性组合)、AIBrix(过滤式)。

第二类是基于预测的未来指标类方法。这类方法尝试通过对性能指标进行预测(例如TTFT、TPOT)来指导请求调度决策,根据请求在对应实例的预测推理时延进行路由。这样的策略目标是对请求执行性能进行选择,请求执行性能同时也能包含实例负载、前缀匹配的因素,从而间接达到兼顾集群负载均衡的目标。代表系统包括llm-d(TTFT预测)和Mooncake(以KVCache为中心的调度)。但已有的基于预测性能的服务系统并未说明如何实现预测的能力。通过模拟器进行性能预测是一个直接、自然的想法,但基于模拟器实现性能预测调度策略的过程中存在很多问题未被阐明。

\subsection{从模拟器到在线预测的关键问题}

将离线模拟器应用于在线调度场景面临三个关键问题。

语义不同。离线模拟器通常输入完整workload,通过事件循环推进系统状态并输出整体性能指标;而在线调度需要在每个请求到达时,对多个候选实例进行反事实性能预测,返回用于比较的关键指标。两者的输入输出语义存在本质差异。

状态同步。在线系统是分布式运行的,组件间通信和各自行为引入执行的不确定性。全局调度器对推理实例的信息观测存在时延,推理实例内部的Tokenize过程对先后到达的请求不保序,这些因素可能导致模拟器维护的状态与真实实例状态发生偏移。若不处理状态同步问题,基于模拟器的预测将逐渐失真,进而影响调度决策质量。

开销预算。全量复刻实例内部调度与KVCache管理会带来高昂的预测开销,难以满足在线调度路径对吞吐与延迟的要求。因此在线模拟器需要在精细复刻与在线可用之间做出权衡,并通过同步机制保证预测长期有效。

\section{本文工作与主要贡献}

本文面向LLM推理集群的全局调度场景,尝试通过性能模拟器实现在线请求性能预估,解决实现过程中的状态同步与开销问题,并系统分析预测能力对调度决策效果的影响。具体来说,在PD共置部署形态下设计在线模拟器的语义、接口与架构,通过状态简化和Prediction Reuse大幅降低预测时延,使基于预测的调度策略能够稳定运行在在线路径中。在此基础上,本文通过端到端实验对比基于模拟器的策略与多项已有调度策略,分析预测准确性与吞吐对服务质量的影响。

主要贡献总结如下:

\begin{enumerate}
  \item \textbf{PD共置场景下的预测能力设计。} 给出在线模拟器在全局调度场景中的语义与接口定义,提出面向PD共置的模拟器架构设计,包括请求写入、性能预测与状态同步机制。分析在线场景中导致模拟状态偏移的原因(观测时延与Tokenize不保序),为在线预测的可用性提供保证。

  \item \textbf{面向在线调度的优化与简化。} 首先实现full-fidelity参考模拟器以对齐实例语义,但发现其预测吞吐难以满足在线调度需求。随后通过状态简化(将模拟粒度从token-level提升至batch-pattern level)和Prediction Reuse(复用预测推进结果),显著降低预测开销。优化后的预测平均时延控制在50微秒以下,完全满足在线路径吞吐需求,不成为系统瓶颈。

  \item \textbf{实验评估与影响因素分析。} 在8卡NVIDIA H20与8卡NVIDIA H100集群上进行端到端测试,模型覆盖Qwen2.5-7B(Dense)和Qwen3-30B-A3B(MoE),工作负载包括Qwen Chatbot、Qwen Coder、Qwen Agent和Kimi ToolAgent。对比基于模拟器的策略与vLLM-v1、NVIDIA Dynamo、Company-X、llm-d等调度策略,验证性能提升效果。分析预测准确性与吞吐对端到端服务质量(TTFT/TPOT、吞吐、尾延迟)的影响,阐明模拟器为何能在在线调度中起到帮助。
\end{enumerate}
% !TEX root = ../main.tex

\chapter{背景与动机}
\label{ch:background}

\section{LLM推理基础}

\subsection{Prefill与Decode阶段}

大语言模型以自回归方式生成文本。给定输入提示(prompt),模型逐token产生输出,直到生成终止符或达到最大长度。这一过程可明确划分为两个阶段。

Prefill阶段处理输入提示中的所有token,并行计算得到第一个输出token。由于输入token可并行处理,Prefill阶段属于计算密集型操作,其执行时间主要取决于输入长度和模型参数量。完成Prefill后,输入提示的键值缓存(Key-Value Cache,简称KVCache)被保存在GPU内存中,供后续阶段复用。

Decode阶段逐token生成剩余输出。每生成一个新token,模型需要将该token作为输入执行一次前向传播,并更新KVCache。Decode阶段属于访存密集型操作,其执行时间主要受显存带宽和批次大小影响,与输出长度呈线性关系。

两个阶段的性能指标直接决定用户体验。时间到首token(Time To First Token,TTFT)衡量从请求到达至返回第一个输出token的耗时,反映系统的响应速度。时间到每token(Time Per Output Token,TPOT)衡量生成后续token的平均间隔,反映系统的流式输出流畅度。交互型应用如对话系统和代码助手对这两项指标均有严格要求。

\subsection{KV Cache机制}

KVCache是加速LLM推理的关键技术。Prefill阶段计算得到的键矩阵和值矩阵被缓存下来,Decode阶段只需计算当前token的键值,并与历史缓存拼接后参与注意力计算。这避免了每步重新计算历史token的键值,大幅降低计算量。

KVCache还带来一个重要特性:缓存命中可加速Prefill阶段。若新请求的输入前缀与某已完成请求完全匹配,且该请求的KVCache仍保留在内存中,则实例可直接复用缓存的前缀计算结果,仅需对剩余输入token执行Prefill。这称为前缀缓存命中。缓存命中的请求Prefill耗时显著降低,TTFT随之改善。

缓存命中率受多种因素影响:请求前缀的重复模式、缓存容量、缓存淘汰策略、请求路由决策等。其中路由决策由全局调度器控制,将共享相同前缀的请求导向同一实例可提高命中率,但也可能导致该实例负载过重。

\subsection{PD共置下的混合批处理}

在PD共置部署形态中,Prefill与Decode请求在同一实例的同一GPU资源池上混合执行。实例调度器在每个迭代步中从队列中选择一批请求执行,这批请求可能同时包含Prefill阶段的新请求和Decode阶段的进行中请求,形成混合批(mixed batch)。

混合批处理中两阶段存在复杂的相互影响。Prefill请求计算密集,需要访问模型参数并计算完整前向传播;Decode请求访存密集,主要从KVCache读取历史状态并计算当前token。两者在GPU上竞争计算资源(SM)和显存带宽。当混合批中Prefill请求占比较高时,Decode请求的TPOT可能显著增加;反之,大量Decode请求也会延长Prefill请求的等待时间。

现代推理引擎采用分块预填充(chunked prefill)来缓解这种相互影响。长Prefill请求被切分为多个小块,与Decode请求交错执行,避免单个大Prefill阻塞Decode过久。但分块预填充增加了调度复杂性:每个Prefill块何时执行、与哪些Decode请求组成批次,都会影响端到端延迟。

这种复杂性使得PD共置场景下的性能预测远比PD分离场景困难。PD分离将两阶段隔离开,Prefill池和Decode池的负载可分别用独立指标刻画。PD共置下,实例负载与TTFT/TPOT的关联难以用简单指标(如队列长度、batch size)准确描述,因为同一batch size下,Prefill与Decode的不同配比会产生截然不同的执行时间。这为全局调度带来了挑战:调度器需要更精细的手段来预估请求在不同实例上的实际执行性能。

\section{集群级全局调度}

\subsection{系统架构与信息流}

LLM服务集群由全局调度器和多个推理实例组成。调度器作为集群入口,接收所有外部请求,并为每个请求选择目标实例。每个实例独立维护请求队列、运行集合和KVCache,执行本地调度与批处理。

调度器依赖实例上报的状态信息进行决策。常见上报指标包括:队列长度(Q-Len)、运行请求数(R-BS)、总token数、GPU利用率等。这些指标通过心跳或响应头携带的方式从实例传递至调度器。由于网络传输和实例处理存在时延,调度器观测到的状态总是滞后于实例真实状态。此外,实例内部的Tokenize过程对先后到达的请求不保证处理顺序,可能导致调度器维护的请求顺序视图与实例实际执行顺序不一致。这种观测时延和顺序不确定性构成状态同步问题的根源。

\subsection{调度决策的两大目标}

全局调度器需要在请求到达时快速做出路由决策。有效调度追求两个有时相互冲突的目标。

负载均衡旨在避免部分实例过载而其他实例空闲。过载实例的请求排队时间增加,TTFT和TPOT随之恶化;空闲实例则浪费宝贵GPU资源。传统负载均衡策略(如Join the Shortest Queue)在LLM服务中仍有应用,但需适配LLM特有的计算特征。

缓存感知旨在利用KVCache前缀命中加速Prefill。将共享相同前缀的请求路由至同一实例可提高缓存命中率,降低TTFT。但过度追求缓存命中会导致请求堆积在少数实例上,破坏负载均衡。如何在两者间取得权衡是调度策略设计的核心难点。

\subsection{调度性能的度量}

端到端服务质量最终体现为TTFT和TPOT的分布特性。除平均值外,尾延迟(如P95、P99)对用户体验同样关键。少数请求的异常延迟可能破坏用户对整体服务的印象。

调度策略不直接产生计算延迟,而是通过影响实例负载分布和缓存命中率来间接影响端到端指标。因此评估调度策略需要端到端实验:在真实硬件上重放请求trace,测量不同策略下的TTFT/TPOT分布。单纯的模拟或理论分析不足以揭示调度策略在实际系统中的表现。

\section{现有调度策略分析}

\subsection{对当前状态Metrics的混合}

第一类方法基于实例当前的服务指标进行加权组合或过滤,直接生成调度决策。这类方法实现简单,部署广泛,但存在一些局限。

\textbf{vLLM默认策略}采用负载均衡优先的设计。调度器计算每个实例的得分:score = 4 × Q-BS + 1 × R-BS,其中Q-BS为队列中等待的请求数,R-BS为正在运行的请求数。得分最小的实例被选中。该策略完全忽略KVCache命中的影响,在缓存命中率高的场景中,这种策略会错失大量加速机会。

\textbf{ai-Dynamo与Company-X}采用线性组合策略。调度器维护两个指标:缓存命中率(或命中节省的Prefill token数)和实例负载(batch size或总token数)。得分计算为二者的加权和,通过调整权重控制两个目标的侧重。这类策略能够兼顾缓存感知和负载均衡,但权重需要针对特定workload调优。

\textbf{AIBrix}采用过滤式组合。调度器先检查集群负载是否失衡,若负载差异超过阈值则优先负载均衡,否则基于缓存命中率选实例。阈值同样需要人工设定。

\textbf{动机实验:Company-X调参敏感性分析}为说明线性组合方法的局限,我们在Qwen-Chat workload上改变Company-X的权重参数进行实验。实验固定硬件为8×H20,模型为Qwen3-30B-A3B,请求trace为20分钟真实对话请求。权重参数λ从0.3变化至0.9(λ越靠近0越偏向负载均衡,越靠近1越偏向缓存感知)。实验发现:最优λ值随workload变化,对话场景最优λ为0.7,代码场景最优λ为0.5,智能体场景最优λ为0.6。同一场景下,λ偏离最优值0.1导致平均TTFT增加15\%-25\%。这表明静态调参难以适应workload变化,而动态调整又缺乏明确指导原则。

\subsection{对未来状态或性能Metrics的预测}

第二类方法尝试预测请求指派后的执行性能,以预测值作为调度依据。这类方法不直接组合指标,而是通过模型或模拟器预估请求在各实例上的TTFT,选择预估最优的实例。

\textbf{llm-d}是代表性工作。调度器维护每个实例的KVCache状态和队列信息,对新到达请求,预估其在各实例上的TTFT(考虑缓存命中可能带来的加速),选择预估TTFT最小的实例。预估基于简化的性能模型,考虑输入长度、缓存命中长度、队列中等待请求等因素。

\textbf{Mooncake}以KVCache为中心设计调度。调度器将请求按前缀分类,优先将请求路由至缓存该前缀的实例;若这些实例负载过重,则考虑将请求发往无缓存的实例并接受重计算开销。Mooncake同样需要预估不同选择的代价,但其核心是缓存管理而非通用性能预测。

预测类方法的优势在于:预估的TTFT天然融合了负载均衡和缓存感知两个目标。负载重的实例即使缓存命中率高,其预估TTFT也可能因排队而变高;缓存命中率低的实例即使负载轻,预估Prefill时间也可能较长。调度器只需比较单一数值即可完成决策,无需人工设定权重或阈值。

但预测类方法面临实现层面的挑战。如何在在线场景中快速获得准确预估?如何保证预估使用的实例状态与真实状态保持一致?预测不准确会带来多大影响?这些问题在llm-d和Mooncake的论文中未被深入讨论。

\subsection{两类方法的根本差异}

两类方法的差异本质在于决策依据的时间维度。当前Metrics方法依赖请求到达时刻的瞬时状态,这些状态是历史积累的结果,但未必能准确反映请求未来的执行体验。预测类方法尝试预估请求的未来执行情况,直接面向决策目标,但需要付出额外的计算和同步成本。

从信息利用角度看,当前Metrics方法丢弃了大量信息:batch size相同但Prefill/Decode配比不同的两个实例被同等对待,尽管它们的服务能力可能相差甚远。预测类方法通过模拟或建模,尽可能保留和执行相关的信息,但信息保留越完整,计算开销越高。

\section{现有方法未解决的问题}

现有工作虽已尝试将性能预测引入调度决策,但三个关键问题仍未得到清晰解答。

第一,在线场景中如何实现可用的性能预测?预测依赖实例状态,但分布式系统固有的观测时延和处理顺序不确定性导致状态同步困难。调度器维护的模拟状态如何与真实实例状态保持一致?预测器如何设计才能在在线路径中稳定运行?这些问题在llm-d等工作中未被阐明。

第二,预测能力能带来多大的调度性能提升?已有工作未进行端到端对比实验,验证基于预测的调度与经典调度策略的实际差距。在真实硬件、真实workload上,预测类方法相比vLLM、ai-Dynamo等能否带来可量化的TTFT/TPOT改善?

第三,预测质量如何影响调度效果?预测不可能完全准确,但预测误差在多大范围内可接受?预测吞吐是否成为系统瓶颈?预测准确性和吞吐之间存在权衡,如何取舍才能最大化端到端服务质量?

本文面向PD共置部署形态,尝试填补上述空白。我们设计并实现在线模拟器,解决状态同步与开销控制问题;通过端到端实验量化预测类方法的性能提升;系统分析预测准确性和吞吐对调度效果的影响,为推理集群调度设计提供实证依据。
% !TEX root = ../main.tex

\chapter{在线模拟器设计}
\label{ch:design}

离线模拟器是LLM服务系统设计与容量规划的重要工具。给定完整的请求trace和系统配置,
离线模拟器在CPU上步进执行,输出吞吐、时延等性能指标,以评估不同配置的效果。
这类模拟器的运行环境是单机的,状态确定性的线性发展,没有实时性要求。
而在线调度场景的性能预测能力对模拟器提出了完全不同的需求,
模拟器需要在每个请求到达时,快速得到请求在当前实例的执行性能,
即回答“如果将当前请求发往实例i,它的TTFT会是多少?”这样的反事实问题。
这要求模拟器以在线服务的方式运行,与推理实例分布式部署和状态同步,
对请求预测能够实时响应,并达到微秒级预测延迟。

将离线模拟器直接用于在线调度面临三个根本性差异。

第一,接口和语义不同。离线模拟器通常提供批处理接口:
输入trace文件,输出性能报告。而在线调度需要的是低延迟的查询接口,
能够为每个到达的请求并行查询多个实例,并在微秒级返回预测结果。

第二,状态维护不同。离线模拟器中,模拟器完全决定实例状态,
所有信息都在单进程内同步更新。在线场景中,调度器和实例是分布式部署的,
并且实例调度器为了优化吞吐、公平性等目标,
正在采用越来越复杂的设计——优先级抢占、动态批处理、分块预填充、请求优先级队列。
这些行为为调度引入了不确定性和复杂性,也为模拟器状态同步带来了挑战。

第三,执行过程不同。在线调度需要回答的是“如果把这个请求发过去会怎样”。
这是反事实的执行过程,需要在当前已经发生的状态之上,叠加一个尚未发生的请求,
向前模拟直到该请求产出首token。
这要求模拟器具备“快照”能力:能够基于当前状态创建假设分支,
在不影响真实状态的情况下执行模拟。

\section{状态同步机制}

忽略由生成Token的不同带来的随机性,常见的实例调度器批处理行为通常是确定的,即在相同的状态下能够构造出完全相同的批,从而得到相近的执行时间,这是模拟器能够对LLM实例或集群推理性能进行相对准确预测的前提。

在离线模拟器中,通过在给定Workload中包含请求生成长度、生成Token信息和请求到达时间信息来排除请求到达和生成Token的随机性。
在预测过程中,由模拟器完全决定实例状态。具体来说,执行队列、等待队列和KVCache等状态由模拟器管理,
并随事件循环的时间推进而模拟调度执行和状态更新。这意味这状态管理是确定性线性推进的。
而在在线场景中,全局调度器和推理实例是分离的组件,由全局调度器维护每个实例的状态镜像,
为了保证性能预测的准确,需要保证两者基于相同的状态。

\subsection{状态同步的必要性}
推理实例根据动态生成的Token控制请求终止(例如EOF或用户指定的终止字符),
推理实例比全局调度器包含更多确定性信息,模拟器需要同步推理实例的最新状态来消除不确定性。

并且实例调度器正在变得越来越智能,一些研究工作引入了优先级抢占、请求暂停与恢复、公平性队列等更复杂的行为。
这些行为在实例内部是确定性的,但从外部观察,全局调度器很难仅凭自管的信息来还原实例的完整状态和行为。

由于上述在线场景下全局调度器的信息缺失和职责不同,依赖全局调度器去猜测实例的状态变化是不可行的。
需要让实例主动将状态变更和执行行为上报给调度器以实现状态同步机制。

\subsection{同步接口设计}

常见的推理框架通过OpenAI接口接收推理请求,但全局调度器不便于通过OpenAI接口返回的请求信息得知推理实例的执行行为。
因此为推理框架添加新的SSE长连接接口,启动时全局调度器与每个实例建立连接。
在推理实例每次发生影响未来调度的状态变化时,将调度结果、执行结果的关键信息上报给全局调度器。

推理实例状态变化在两个时间点发生:第一是接收到全局调度器指派的新请求并加入请求队列,此时只需要在模拟器中也将新请求添加到请求队列。
第二是推理实例每次批处理结束时,此时需要更为精细的状态上报与更新。
状态上报的设计需要权衡信息量和通信开销。对于完整的状态快照,因为包含所有KVCache的分配信息,信息量由于过于昂贵不可接受。
本文采用增量更新上报的方式:上报的信息只包含自上次上报以来发生变化的部分。
例如,当前批处理包含的请求ID和关键信息(处理长度、新生成的Token以及是否完成)、
KVCache的变化(以vLLM为例,包括新分配的Block ID、对应的Hash和剔除的Block ID)以及特殊事件的记录例如请求抢占等。

而对于全局调度器,则需要根据上报信息更新每个实例的状态镜像,维护请求队列和KVCache状态的镜像。
\subsection{同步机制的代价}
对于推理框架添加同步机制,带来的额外处理和网络开销非常小。
这是因为摘要体积很小,通常只有几十K字节,在推理实例调度和批处理结束状态更新时插桩,对网络和实例处理的开销可以忽略(TODO! 可以测一下实例处理的开销)。
并且状态上报的频率和推理实例的批处理频率相当,以实验的H20机器为例,根据批大小和请求上下文长度不同,上报间隔通常在几十毫秒到几百毫秒之间。
在全局调度器上的平均处理时延为(TODO!),完全能够满足在线调度的需求,不会阻塞调度决策。

状态同步对实例的改动也很小,主要是状态变更点的埋点和添加状态上报的接口,在vLLM中实现状态同步增加约x行代码,。

\section{状态演进管理}
与离线的模拟器不同,在线模拟器的状态存在反事实的预测过程。
每次收到新请求时,模拟器会模拟这个新请求在实例状态上的调度,这会导致实例状态如请求队列、KVCache的改变。
在这个过程并不应影响模拟器维护的真实状态,因为如果该请求没有被指派到该实例,后续请求的预测从正确的状态开始。
因此需要对预测过程中的状态演进进行管理,保证不会影响到真实状态。

\subsection{直接的做法: 独立预测}

最简单直接的做法是:每次收到新请求,都从当前状态S开始,克隆出一份请求队列作为预测状态,完整模拟新请求完成。
如(TODO!)所示,考虑连续到达的三个请求r1、r2、r3。
假设r1到达时状态为S0,我们模拟得到它的TTFT,同时模拟过程会从S0推进到S1(经过若干步)。
如果r2到达时实例状态仍然是S0(r1还在排队,尚未开始执行),按照独立预测的做法,我们要从S0开始模拟r2的执行。
但从S0推进到S1的过程实际上被重复计算了,r1之后若干步的模拟结果本可以被r2复用。

这引出一个关键洞察:当前常见的推理框架的请求调度是遵循先到先服务的,新到达请求需要排在当前请求队列的末尾,
因此新请求的执行状态演进路径必然包含前序未完成请求的演进路径,这为预测复用提供了可能。
预测状态管理应该设计为状态演进式的,即维护一个最新的演进状态,新请求在上一次预测推进后的状态基础上开始预测。

因此每个实例可以维护一个演进状态和批处理队列:演进状态作为对模拟器最新状态的预测,新请求可以直接在演进状态上克隆请求队列开始性能模拟,称为模拟状态,并在模拟调度时将每个得到的批压入批处理队列。
如果请求确认指派到该实例,那么模拟状态就可以作为演进状态并保留批处理队列,实现预测复用。
如果请求没有指派到该实例,那么将模拟状态丢弃,保持演进状态不变,等待下一个请求到来时继续在演进状态上模拟。

预测过程如(TODO!),在调用\texttt{predict(r)}时,在当前模拟状态上插入r并向前推进,将每一步推进后的状态压栈,累加各步的执行时间得到TTFT。
请求确认指派时将模拟状态作为新的演进状态并保留批处理队列,供后续请求复用。


\subsection{与状态同步的配合}

状态演进式预测依赖于一个假设:预测的状态演进路径,与实例真实的状态演进路径是一致的。但现实中,实例的真实执行可能因为各种原因与预测不同:
实例内部对到达请求进行并行分词,由于分词时长差异导致调度顺序被改变;或者发生请求终止时被推理实例从请求队列中剔除;以及在状态同步章节(TODO! 插入标签)提到的非确定性调度行为等。

通过状态同步机制在这里校准演进状态。当调度器收到推理实例批处理完成后的状态同步时,在\texttt{sync}中与预测批处理队列的头部批进行比对:
如果真实状态与预测状态一致,说明预测是正确的,将栈顶弹出,预测状态与真实状态对齐。
如果真实状态与预测状态不一致,说明预测路径与真实路径发生了偏离。此时清空整个演进状态和批处理队列,将演进状态重置为真实状态。
(TODO 这里给出偏离情况,并讲解偏离时如何更新演进状态)

这种设计将预测视为“提前执行”的模拟,而同步则负责在真实执行发生时进行校准。只要预测与真实的偏离不频繁,预测状态就能保持复用,保证预测的有效性。


\section{语义与架构设计}

在这一部分分析在线模拟器与离线模拟器的语义不同,暴露三个接口,可以对Predict和AddRequest进行设计
此外,讲解架构设计,全局调度器中包含模拟器模块,实例侧包含状态同步模块,二者通过RPC通信实现状态同步和预测查询。

\subsection{接口}

\begin{verbatim}
fn predict(&self, req: &Request, instance_id: InstanceId) -> Result<Duration>
\end{verbatim}

\texttt{predict}是核心预测接口。给定请求r和目标实例i,返回请求r如果发往实例i的预估TTFT。

实现逻辑:
\begin{itemize}
  \item 获取实例i当前维护的预测状态(即上一次预测推进后的最新状态)。
  \item 在预测状态上插入请求r,按照3.4节的演进规则向前模拟,直到r的prefill完成。
  \item 累加各步的执行时间,返回总时间。
  \item 将模拟过程中产生的中间状态(每一步推进后的batch pattern)压入预测状态栈,供后续请求复用。
\end{itemize}

由于调度器需要同时为多个实例进行预测,\texttt{predict}支持并发调用,内部实现无锁,利用Rust的异步运行时并行执行。

\subsection{add\_request}

\begin{verbatim}
fn add_request(&self, req: &Request, instance_id: InstanceId)
\end{verbatim}

\texttt{add\_request}在调度器确认将请求r发往实例i后调用。这个接口的作用是更新模拟状态,让模拟状态“知道”请求r确实已经到达实例i。

为什么需要这个接口?因为\texttt{predict}是反事实的——它预测“如果把请求发过去会怎样”,但请求可能最终没有被发往这个实例。\texttt{add\_request}负责将“反事实”变为“事实”:将请求r正式加入实例i的模拟状态队列。

实现逻辑很简单:将请求r加入实例i的等待队列统计信息中。如果实例i的预测状态栈非空(即已经有一些预测推进),\texttt{add\_request}会将请求r加入最新状态,并从那里开始继续演进。

\texttt{add\_request}与\texttt{predict}配合使用:先\texttt{predict}所有候选实例,选择最优目标,然后对选中的实例调用\texttt{add\_request}确认,对其他实例不做操作(它们的模拟状态保持不变)。

\subsection{sync}

\begin{verbatim}
fn sync(&self, instance_id: InstanceId, state_delta: StateDelta) -> Result<()>
\end{verbatim}

\texttt{sync}由实例回调调用,用于将真实状态变更同步给调度器。\texttt{StateDelta}包含版本号和变更摘要,具体格式在3.2节已描述。

实现逻辑:
\begin{itemize}
  \item 根据版本号判断同步信息的有效性。
  \item 如果版本号连续,将变更摘要应用到模拟状态上。
  \item 同时,将变更摘要与预测状态栈的栈顶进行比较:
    \begin{itemize}
      \item 如果栈顶状态与真实状态一致,弹出栈顶,预测状态与真实状态对齐。
      \item 如果不一致,清空整个预测状态栈,将预测状态重置为真实状态。
    \end{itemize}
\end{itemize}

\texttt{sync}是状态演进式预测能够长期保持有效性的关键。它确保预测不会偏离真实太远,一旦偏离就被及时纠正。


下一章将通过端到端实验,验证本章设计在实际系统中的效果。
% !TEX root = ../main.tex

\chapter{总结}


\section{主要结论}
本文在已有的训练、推理模拟器工作和基于时延预测的请求调度策略基础上,
在全局调度器中以独立的模块实现在线推理模拟器,为调度决策提供时延预测能力,在端到端实验中证明对TTFT的预测准确度达到(TODO!)。
为了满足大规模集群部署下扩展性的需求,本文使用基于状态机的状态抽象和预测缓存复用机制来优化预测模块性能,
大幅提升了预测吞吐和请求预测时延,在128实例的扩展性测试中也不会成为集群服务能力瓶颈。
\par 此外,本文在统一的集群推理框架中实现了当前工业界主流推理框架(TODO!)和学术界高影响力的工作(TODO!)中的调度策略,
通过端到端的数据集重放,公平的比较了这些策略在不同数据集、不同硬件环境、不同服务模型下的请求服务指标。
实验表明,基于模拟器的预测方式能够提高集群的服务能力
\section{研究展望}
虽然本文实现了用于请求调度决策的推理性能模拟器,并达到了(TODO!)的准确度,但是由于模拟器对执行时长建模的不足、以及分布式系统组件之间的固有缺陷,
预测的准确性仍然有较大的提升空间,



}

%TC:ignore

\clearpage
{
\ExplSyntaxOn
\bool_if:NTF \g__sjtu_twoside_bool
{
    \fancyhead [ LE ]     { 参考文献 }
    \fancyhead [ RO ]     { 参考文献 }
}
{
    \fancyhead [ R ] { 参考文献 }
}
\ExplSyntaxOff
% 文献表字体
\renewcommand{\bibfont}{\zihao{5}}
% 设定固定间距
\fixedlineskip{15.6bp}
{
\ctexset{chapter={afterskip=26bp}}
% 参考文献
\printbibliography[heading=bibintoc]
}
\clearpage
}

\makeatletter
% \appendix采用数字编号。
\renewcommand{\appendix}{\par
    \setcounter{chapter}{0}
    \setcounter{section}{0}
    \ctexset{chapter/number={\arabic{chapter}}}
}
% 使用 \appchapter 替代附录中的 \chapter 章节,附录中的章节不再放入目录。
\newcommand{\appchapter}[1]{
    \refstepcounter{chapter}
    \SJTU@head*[附录 \thechapter]{#1(附录 \thechapter)}
}
\makeatother

{
\ctexset{chapter={afterskip=26bp}}
% 附录
\appendix
% 附录中图表不加入索引
\captionsetup{list=no}
% !TEX root = ../main.tex

\appchapter{符号与标记}\label{chap:symbol}
\addcontentsline{toc}{chapter}{附\quad 录}  

}


% 结尾部分
\backmatter

{
\ctexset{chapter={afterskip=26bp}}
% 发表论文及科研成果
% !TEX root = ../main.tex

\begin{achievements}

\begin{bibliolist}{00}
  \item 张三,李四. …… (已录用)
  
\end{bibliolist}


\end{achievements}

}

\clearpage
{
\ExplSyntaxOn
\bool_if:NTF \g__sjtu_twoside_bool
{
    \fancyhead [ LE ]     { 致谢 }
    \fancyhead [ RO ]     { 致谢 }
}
{
    \fancyhead [ R ] { 致谢 }
}
\ExplSyntaxOff
{
\ctexset{chapter={afterskip=26bp}}
% 致谢
% !TEX root = ../main.tex

\begin{acknowledgements}
在本文的撰写过程中,我的导师给予了莫大的帮助,也非常感谢实验室的学长学姐,在论文撰写过程中提供指导。
此外,我还要感谢提供支持和鼓励的同学们,
\par 感谢在大学的四年成长期间提供指导、帮助的各位老师,感谢阿里通义实验室提供的计算资源、真实数据和真实服务场景下的落地机会。
最后,我要向坚持不懈、在迷茫中探索的自己也献上诚恳的致谢。
\end{acknowledgements}

}

\clearpage
}

{
\ctexset{chapter={afterskip=26bp}}
% 学士学位论文要求在最后有一个大摘要,单独编页码
% !TEX root = ../main.tex

\begin{digest}
  HCCI (Homogenous Charge Compression Ignition) combustion has advantages in terms of efficiency and reduced emission. HCCI combustion can not only ensure both the high economic and dynamic quality of the engine, but also efficiently reduce the NOx and smoke emission. Moreover, one of the remarkable characteristics of HCCI combustion is that the ignition and combustion process are controlled by the chemical kinetics, so the HCCI ignition time can vary significantly with the changes of engine configuration parameters and operating conditions. ……
\end{digest}

}

\end{document}
