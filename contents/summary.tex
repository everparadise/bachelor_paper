% !TEX root = ../main.tex

\chapter{总结}


\section{主要结论}
本文在已有的训练、推理模拟器工作和基于时延预测的请求调度策略基础上,
在全局调度器中以独立的模块实现在线推理模拟器,为调度决策提供时延预测能力,在端到端实验中证明对TTFT的预测准确度达到(TODO!)。
为了满足大规模集群部署下扩展性的需求,本文使用基于状态机的状态抽象和预测缓存复用机制来优化预测模块性能,
大幅提升了预测吞吐和请求预测时延,在128实例的扩展性测试中也不会成为集群服务能力瓶颈。
\par 此外,本文在统一的集群推理框架中实现了当前工业界主流推理框架(TODO!)和学术界高影响力的工作(TODO!)中的调度策略,
通过端到端的数据集重放,公平的比较了这些策略在不同数据集、不同硬件环境、不同服务模型下的请求服务指标。
实验表明,基于模拟器的预测方式能够提高集群的服务能力
\section{研究展望}
虽然本文实现了用于请求调度决策的推理性能模拟器,并达到了(TODO!)的准确度,但是由于模拟器对执行时长建模的不足、以及分布式系统组件之间的固有缺陷,
预测的准确性仍然有较大的提升空间,


