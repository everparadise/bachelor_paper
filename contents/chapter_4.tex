%%%%%%%%%%%%%%%%%%%%%%%%%%%%%%%%%%%%%%%%%%%%%%%%%%%%%%%%%%%%%%%%%%%%%%%%%%%%
% 第四章:实现与优化(从不可行到可用)
%%%%%%%%%%%%%%%%%%%%%%%%%%%%%%%%%%%%%%%%%%%%%%%%%%%%%%%%%%%%%%%%%%%%%%%%%%%%
\chapter{实现与优化:从不可行到可用}
\label{ch:sim_impl}

本章介绍在线推理模拟器的实现细节与关键优化过程。
我们的实现目标是:在不显著增加在线调度关键路径延迟的前提下,
为每个到达请求提供多实例 TTFT 预测信号,并通过同步机制维持预测有效性。
实践表明,直接将离线模拟器的 full-fidelity 设计搬到在线路径会导致预测吞吐不足;
因此我们采用轻量化状态抽象与预测复用等方法,使预测模块能够满足在线调度需求。

\section{实现概述}
\label{sec:impl_overview}

在线模拟器部署在全局调度器侧,为每个推理实例维护一份模拟状态副本。
当请求到达时,全局调度器并行调用多个实例模拟器的 \texttt{Predict} 得到 $\widehat{\mathrm{TTFT}}$ 等指标并完成路由决策;
路由完成后通过 \texttt{EnqueueRequest} 写入目标实例状态。
同时,实例调度器在每个 batch 完成后上报 \texttt{SyncBatchResult},用于修正 $S^{gt}$ 并更新模式信息。

本文实现以 PD 共置 mixed batch 场景为主要目标:Prefill 与 Decode 在同一实例混合执行,TTFT 对 batch 模式和队列演化敏感。
因此在线预测模块不仅要快,还要能在状态漂移存在时通过同步保持可用。

\section{Full-fidelity 参考实现与瓶颈}
\label{sec:full_fidelity_ref}

\subsection{设计动机:语义对齐与开销归因}
\label{sec:ref_motivation}

我们首先实现一个 full-fidelity 的参考模拟器(reference implementation),尽可能复刻运行时(如 vLLM)的关键行为:
(1) KVCache block 级分配与回收(allocate/evict);
(2) request-level 的 waiting/running 队列维护;
(3) 与运行时一致的实例调度逻辑,用于逐步构造 batch;
(4) 每次预测从当前 $S^{gt}$ fork 出独立状态进行反事实推进,避免污染基准状态。

该参考实现用于验证模拟器语义正确性与定位开销来源,
\textbf{但并不作为后续端到端调度对比的 baseline}。
这是因为其设计目标偏向“精细复刻”,而在线调度路径需要“低开销可用”的预测信号。

\subsection{瓶颈分析:在线不可用}
\label{sec:ref_bottleneck}

实验表明,full-fidelity 参考实现的预测吞吐难以满足在线调度需求(TODO: 给出测得 QPS 与在线需求对比)。
主要原因包括:

\textbf{(1) 预测开销与 backlog 强相关。}
request-level 构批需要遍历 waiting/running 队列并模拟多步推进,
当实例 backlog 堆积时,每次预测会重复执行大量相同的队列扫描与调度逻辑,导致计算开销显著上升。

\textbf{(2) KVCache 细粒度管理常数开销大。}
block 级分配/回收涉及复杂数据结构与一致性维护,
在在线预测中复刻该逻辑会引入较高常数开销,与“毫秒级在线预算”不匹配。

\textbf{(3) fork 与状态复制成本。}
为确保多候选预测相互独立,参考实现每次 \texttt{Predict} 都会 fork 完整状态,
在高并发预测调用下,该复制成本进一步放大。

这些现象表明:若坚持 full-fidelity 复刻,在线预测将成为调度关键路径瓶颈。
因此,我们需要在“复刻精度”与“在线可用性”之间做出权衡,并通过同步闭环控制误差。

\section{轻量实现:runtime-assisted + pattern-level scheduling}
\label{sec:lightweight_impl}

\subsection{设计原则}
\label{sec:lightweight_principles}

轻量实现围绕两个原则展开:

\textbf{原则1:将与运行时强耦合、常数开销大的细节外包给运行时上报。}
典型代表是 KVCache block 级管理:由运行时维护最准确,也最节省模拟器开销。
模拟器只消费必要的摘要信息与批次模式(pattern),避免在预测路径中复刻复杂逻辑。

\textbf{原则2:将状态粒度提升到 batch/pattern 层面,避免 request-level 遍历构批。}
在 PD 共置 mixed batch 场景下,TTFT 的核心在于“新请求何时进入某个可执行 batch”以及该 batch 的 prefill 负载。
因此我们以最近 batch 的模式与队列摘要来驱动短视窗预测,避免在每次预测时遍历队列并重算 batch 构造过程。

\subsection{pattern-level scheduling 的实现}
\label{sec:pattern_scheduling_impl}

在轻量实现中,实例模拟器维护最近 batch 的 \texttt{pattern}(例如 prefill/decode 长度)与逻辑时钟,
并用查表/性能模型估计下一批的执行耗时(预测器细节本文参考 Vidur,TODO: 简述查表维度)。
当新请求到达时,模拟器不再重构完整 waiting/running 队列,而是在当前模式基础上估计该请求被并入 batch 的时刻与 prefill 负载,
从而得到 $\widehat{\mathrm{TTFT}}$。

该过程的关键是:预测复杂度不再随 backlog 线性增长,而近似为常数(或与少量统计桶数相关,取决于 \texttt{queue\_summary} 的实现,TODO)。
这一设计使预测模块能够在高并发下保持稳定吞吐。

\section{Prediction reuse:减少重复推演}
\label{sec:prediction_reuse}

\subsection{重复计算的来源}
\label{sec:reuse_motivation}

在在线系统中,请求往往密集到达。
若每个新请求到达都从 $S^{gt}$ 出发重新推进预测,会导致对 backlog 的调度与模式演化被反复计算。
特别是在 waiting backlog 较大时,相邻请求的预测共享大量相同的“历史推进轨迹”,全量重算属于重复开销。

\subsection{复用策略}
\label{sec:reuse_method}

为减少重复推演,我们引入 prediction reuse:
对每个实例维护一个“预测前沿状态”(predicted frontier),表示上一次预测推进后的末状态。
当新请求到达时,模拟器从该前沿状态继续推进增量预测,而非从 $S^{gt}$ 重算完整 backlog。
当收到 \texttt{SyncBatchResult} 时,模拟器以实例上报为准修正 $S^{gt}$,
并在必要时重置预测前沿(例如 batch\_id 不连续或差异超过阈值,TODO)。

该策略在不改变最终预测语义的前提下显著减少了重复计算,降低了单次预测延迟与 CPU 开销。

\subsection{收益定位:控制面延迟与鲁棒性}
\label{sec:reuse_value}

需要注意的是,系统端到端吞吐瓶颈可能位于请求分发、网络传输或实例执行等路径,
因此 prediction reuse 不一定在整体 QPS 上体现为线性提升。
但由于候选实例预测通常并行执行,调度决策的关键路径更受最慢预测任务与预测服务排队影响。
prediction reuse 通过降低预测任务的计算开销与抖动,有助于降低调度控制面的尾部延迟,
在突发负载或 CPU 紧张时提升系统鲁棒性(第~\ref{ch:evaluation}~章将给出对比结果,TODO: 对应图表)。

\section{小结:三版本实现对比与实践结论}
\label{sec:impl_summary}

综上,我们以 full-fidelity 参考实现验证语义并定位瓶颈,发现其在线不可用;
随后通过 runtime-assisted 的 KVCache 外包与 pattern-level scheduling 将预测复杂度降至近似常数,实现在线可用的预测吞吐;
在此基础上引入 prediction reuse 进一步减少重复推演,降低预测延迟与抖动。
在实验章节中,我们将展示三种实现的预测性能对比(预测 QPS、预测延迟分位数),并评估其对端到端调度效果的影响(TODO: 引用实验图表)。
