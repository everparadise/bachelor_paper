% !TEX root = ../main.tex

%%%%%%%%%%%%%%%%%%%%%%%%%%%%%%%%%%%%%%%%%%%%%%%%%%%%%%%%%%%%%%%%%%%%%%%%%%%%
% 第五章:实验评估
%%%%%%%%%%%%%%%%%%%%%%%%%%%%%%%%%%%%%%%%%%%%%%%%%%%%%%%%%%%%%%%%%%%%%%%%%%%%
\chapter{实验评估}
\label{ch:evaluation}

本章通过端到端测试比较包含性能预测能力的调度策略与开源社区、学术界与工业界的调度策略在LLM推理服务中的性能表现,来总结本文性能预测模块对调度决策的提升。
在Baseline选择上,本文选取了一些基于当前性能指标进行混合的策略,来展示对于简单性能指标的选择可能无法全面、准确刻画实例负载情况,与本文实现的性能预测模块得到预测性能指标的调度策略进行对比,展示基于性能预测的调度策略能够避免指标选取、参数调优的过程并且能够得到更优性能。
为了能够排除系统实现、架构设计对于端到端性能的影响,本文在Rust实现的统一全局调度框架上来实现各种策略,该框架已通过严格的性能校准验证其正确性和高效性。并且每种策略的实现也进行了校准,以保证策略实现与其在原框架上的语义相同,性能不慢于原框架实现。
* Company-X是世界最大的LLM服务提供商之一在生产环境中使用的策略,其采用简单的线性组合方式混合性能指标(TODO! 如图)。为了展示其最优效果,本文针对每个Workload通过实验将参数调优到了最佳性能。
* vLLM是LLM推理服务中使用最广泛的框架,采用了基于请求数混合的线性组合负载均衡策略。
* Dynamo是NVIDIA公布的推理框架,其调度策略同样基于线性组合,但是使用了与Company-X不同的性能指标,使用的两个性能指标是Prefill Token数和实例中的总Token数,调度器通过将请求指派到这两者混合得分最小的实例上,本文同样通过实验将参数调整到了最优配置。
而基于性能预测的调度策略实现如下
* Predictor使用本文实现的性能预测模块,预测得到请求在实例上执行完Prefill的时延,并选取时延最小的实例进行请求指派

在实验配置上,本文选取8*H20的DGX节点作为测试环境。使用Qwen2.5-7B的Dense模型和Qwen3-30B-A3B的MoE模型,来验证对于不同架构的模型,本文的性能预测能力同样有效。实验使用PD共置的方式部署实例,模型独立部署在一个GPU上,因此得到了8实例的推理服务集群进行端到端请求重放。
在工作负载选取上,使用全面、多样的开源负载来进行请求重放,具体选用的负载如下
\begin{itemize}
  \item Qwen Chatbot trace(TODO)
  \item Qwen Coder trace(TODO)
  \item Qwen Agent / ToolAgent trace(TODO)
\end{itemize}


\section{端到端性能比较}
\label{sec:e2e_results}


\subsection{稳态负载下的 TTFT 与吞吐}
\label{sec:steady}

图~\ref{fig:e2e_ttft} 展示了在稳态到达下,各策略的 TTFT(mean/p95/p99)对比结果(TODO: 补图)。
总体上,预测驱动策略在 PD 共置 mixed batch 场景下能够更准确地选择“更快获得可执行 prefill 机会”的实例,
从而降低 TTFT,尤其在高负载时对 p95/p99 的改善更显著(TODO: 用你的数据填充结论)。

图~\ref{fig:e2e_tput} 展示了系统吞吐(QPS)对比(TODO: 补图/表)。
预测驱动策略在不降低吞吐的前提下改善 TTFT,或在部分场景同时提升吞吐与 TTFT(取决于你的结果,TODO)。

\section{在线可用性:预测模块吞吐与延迟}
\label{sec:online_budget}

\subsection{预测吞吐与延迟统计}
\label{sec:pred_perf}

我们测量预测模块的处理能力,包括:
(1) 单次 \texttt{Predict} 的延迟(mean/p95/p99);
(2) 在并发请求到达下的预测吞吐(predictions/sec 或 requests/sec);
(3) 预测开销在调度关键路径中的占比(若可测,TODO)。

表~\ref{tab:pred_perf} 总结了预测模块性能(TODO: 填数值)。
总体结果显示,轻量在线模拟器(pattern-level + runtime-assisted)能够在高并发下保持稳定吞吐,
不会成为调度瓶颈(TODO: 结合你系统目标 QPS 说明)。

\begin{table}[t]
  \centering
  \caption{预测模块在线性能(TODO:填入不同版本/不同硬件上的 QPS 与延迟)。}
  \label{tab:pred_perf}
  \begin{tabular}{lccc}
    \toprule
    设置 & 预测吞吐(/s) & p95 延迟($\mu$s/ms) & p99 延迟($\mu$s/ms) \\
    \midrule
    V1 Full-fidelity & TODO & TODO & TODO \\
    V3 Lightweight w/o reuse & TODO & TODO & TODO \\
    V2 Lightweight w/ reuse & TODO & TODO & TODO \\
    \bottomrule
  \end{tabular}
\end{table}

\section{消融实验:同步机制与实现优化的必要性}
\label{sec:ablation}

本节回答 Q3:同步闭环与优化选择是否必要?
考虑到本科毕设篇幅,本节提供两类低成本且与贡献紧密相关的消融实验;你可根据已有数据保留其中一类或两类。

\subsection{同步闭环的必要性(建议保留)}
\label{sec:sync_ablation}

我们比较以下两种设置(TODO:根据实现情况调整):
\begin{itemize}
  \item \textbf{Full-Sync:}每个 batch 完成后上报 \texttt{batch\_trace} 并修正 $S^{gt}$(默认)
  \item \textbf{No-Sync:}禁用 \texttt{SyncBatchResult},模拟器仅靠自身推进(或使用固定背景负载),观察漂移
\end{itemize}

图~\ref{fig:sync_drift} 展示预测误差或端到端 TTFT tail 随时间的变化(TODO)。
结果通常表现为:No-Sync 会产生累积漂移,导致预测失真并损害调度收益;Full-Sync 能将漂移控制在可接受范围内。

\subsection{prediction reuse 的影响(建议保留)}
\label{sec:reuse_ablation}

我们在相同调度策略与相同 workload 下,对比 Lightweight w/ reuse(V2)与 w/o reuse(V3):
尽管端到端吞吐可能受其他模块瓶颈限制而近似不变,
prediction reuse 仍可显著降低预测延迟分位数(例如 p99),并减少预测侧 CPU 压力(若可测,TODO)。
图~\ref{fig:reuse_latency} 给出了 V2 与 V3 的预测延迟对比(TODO),用于支持实现章节中的结论。

\section{现象分析(可选):跨预测表的排序一致性}
\label{sec:crosstable}

本节用于解释你已观察到的现象:使用不同模型/不同预测表进行调度,mean TTFT 变化不大,但 p95/p99 差异更明显。
由于在线调度主要依赖“实例间相对优劣”的比较,而非绝对值精确,
我们用 Spearman 秩相关系数衡量两张预测表在候选实例排序上的一致程度。

\subsection{Spearman 秩相关与选择一致率}
\label{sec:spearman}

对于每个请求 $r$,我们记录候选实例集合 $\mathcal{I}$ 上两张预测表输出的实例评分序列 $\{s_A(r,i)\}$ 与 $\{s_B(r,i)\}$(本文默认 $s=\widehat{\mathrm{TTFT}}$)。
将评分转为排序后计算 Spearman 相关系数 $\rho(r)$,并统计全体请求的平均/分位数(TODO: 公式可简述)。
同时,我们计算两张预测表的 Top-1 选择一致率:
\[
\mathrm{Agree@1} = \frac{1}{|R|}\sum_{r\in R} \mathbb{1}\Big[\arg\min_i s_A(r,i) = \arg\min_i s_B(r,i)\Big]
\]
表~\ref{tab:crosstable} 给出统计结果(TODO)。

\begin{table}[t]
  \centering
  \caption{跨预测表一致性分析(TODO:填 Spearman 与 Agree@1;可再给 tail 子集统计)。}
  \label{tab:crosstable}
  \begin{tabular}{lcc}
    \toprule
    指标 & 全体请求 & tail 子集(例如 TTFT top 10\%) \\
    \midrule
    Spearman $\rho$(均值/中位数) & TODO & TODO \\
    Agree@1 & TODO & TODO \\
    \bottomrule
  \end{tabular}
\end{table}

\subsection{讨论:为何 tail 更敏感}
\label{sec:tail_sensitive}

当系统处于拥塞区间或 mixed batch 的竞争更激烈时,
不同预测表之间的绝对误差与排序误差更容易导致“边界实例”发生选择翻转,从而主要体现在 tail(p95/p99)差异上。
这一现象提示:在 PD 共置场景中,预测表的可迁移性对均值可能更鲁棒,但对尾部质量仍需谨慎评估(TODO: 结合你的数据做出结论)。
