% !TEX root = ../main.tex

\chapter{背景与问题定义}
\label{ch:background}

本章介绍 LLM 在线推理服务的执行过程与关键性能指标,并形式化全局调度中的在线预测问题与约束。
本文后续章节将围绕“在在线调度路径中提供可用的 TTFT 预测与 TTFT SLO 违约风险评估”展开。

\section{全局调度问题定义}
\label{sec:scheduling_problem}

\subsection{系统模型}
\label{sec:system_model}

考虑一个由 $N$ 个推理实例组成的集群,实例集合记为 $\mathcal{I} = \{1,2,\dots,N\}$。
请求 $r$ 在时刻 $t_r$ 到达集群入口,包含请求元信息(request meta),例如 prompt token 数 $p_r$、最大生成长度上界 $g_r$、优先级与 SLO 等(视系统而定)。

全局调度器在请求到达后,需要从候选实例集合中选择一个实例 $i \in \mathcal{I}$ 来处理请求。
实例内部的执行由实例调度器决定(例如动态组批、是否抢占等),全局调度器通常无法直接控制实例内每一步的 batch 构造,但可以通过路由决策影响实例负载与排队结构。

\subsection{在线预测驱动的路由决策}
\label{sec:predictive_routing}

本文关注一种预测驱动的调度范式:在请求 $r$ 到达时,全局调度器对若干候选实例进行性能预测,并据此做路由选择。
形式化地,对于每个候选实例 $i$,在线预测模块输出:
\[
\widehat{\mathrm{TTFT}}(r,i), \qquad \widehat{\Pr}[\mathrm{TTFT}(r,i) > \tau]
\]
其中前者为请求路由至实例 $i$ 的 TTFT 预测值,后者为 TTFT 超过阈值 $\tau$ 的违约风险预测(可选)。

在实际系统中,调度器可将上述预测信号组合为一个实例评分函数(越小越好),例如:
\[
\mathrm{score}(r,i) = \widehat{\mathrm{TTFT}}(r,i) + \lambda \cdot \widehat{\Pr}[\mathrm{TTFT}(r,i) > \tau]
\]
其中 $\lambda$ 控制对尾部风险的偏好程度(TODO: 若本文未使用该项,可在实现章节说明默认 $\lambda=0$)。

\noindent
最终路由选择为:
\[
i^\ast = \arg\min_{i \in \mathcal{I}} \mathrm{score}(r,i)
\]

\subsection{在线预测的约束与挑战}
\label{sec:online_constraints}

与离线容量规划不同,在线预测必须满足严格的工程约束:

\textbf{(1) 时延预算}:预测发生在请求到达后的关键路径上。
为了不显著增加 TTFT,预测模块必须在极短时间内返回结果。
同时,调度器通常需要对多个实例并行预测,因此预测模块不仅要单次预测快,还要具备足够吞吐以支撑高并发请求到达。

\textbf{(2) 状态一致性}:在线预测依赖实例当前状态(队列、批处理节奏等)。
然而全局调度器与实例之间是分布式系统,存在通信延迟与乱序,导致调度器侧维护的模拟状态可能与实例真实状态偏离。
若不进行状态同步与修正,预测误差会随时间累积,最终影响调度效果。

\textbf{(3) 运行时行为复杂}:实例调度器可能包含复杂行为(例如抢占、动态组批策略变化等)。
在线预测模块需要在可用性(开销)与表达能力之间权衡:完全复刻运行时调度与 KVCache 管理会带来高昂开销,
而过度简化又会损害预测有效性。本文后续章节将展示一种可同步的轻量在线模拟器设计,
通过批边界同步将预测漂移控制在可接受范围内。

此外,PD 共置的混合批进一步放大了在线预测的必要性。
在 mixed batch 中,TTFT 不仅取决于当前 waiting 队列长度,还与实例内部下一次可执行批的构成有关;
而批构成又受到 decode 运行集合、prefill 到达时机与调度器策略共同影响。
这使得“用简单指标近似负载”在共置场景下更容易失效。
因此,本文选择以在线模拟/预测的方式输出 TTFT 与违约风险信号,为全局调度提供更直接的决策依据。


\section{本章小结}
\label{sec:background_summary}

本章介绍了 LLM 在线推理服务的执行过程(prefill/decode、动态组批)、关键性能指标(TTFT 及其 SLO 违约),
并形式化了全局调度中的在线预测问题:在请求到达时对候选实例输出 TTFT 与违约风险预测,以指导路由决策。
同时,我们总结了在线预测面临的主要约束:严格的时延预算、分布式状态一致性与运行时行为复杂性。
下一章将围绕这些约束,提出在线推理模拟器的接口语义、状态抽象与同步机制设计。
